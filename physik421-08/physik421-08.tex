% Für Seitenformatierung

\documentclass[DIV=15]{scrartcl}

% Zeilenumbrüche

\parindent 0pt
\parskip 6pt

% Für deutsche Buchstaben und Synthax

\usepackage[ngerman]{babel}

% Für Auflistung mit speziellen Aufzählungszeichen

\usepackage{paralist}

% zB für \del, \dif und andere Mathebefehle

\usepackage{amsmath}
\usepackage{commath}
\usepackage{amssymb}

% Für Literatur/bibliography

\usepackage[backend=biber , style=alphabetic , hyperref=true]{biblatex}

% Für \SIunit[]{} und \num in deutschem Stil

\usepackage[output-decimal-marker={,}]{siunitx}
\DeclareSIUnit\clight{\ensuremath{c}}

% Schriftart und encoding

\usepackage[utf8]{inputenc}
% Bitstream charter als default
\usepackage[charter, greekuppercase=italicized]{mathdesign}
% Lato, als sans default
\renewcommand{\sfdefault}{fla}

% Für \sfrac{}{}, also inline-frac

\usepackage{xfrac}

% Für Einbinden von pdf-Grafiken

\usepackage{graphicx}

% Umfließen von Bildern

\usepackage{floatflt}

% Für weitere Farben

\usepackage{color}

% Für Streichen von z.B. $\rightarrow$

\usepackage{centernot}

% Für Befehl \cancel{}

\usepackage{cancel}

% Für Links nach außen und innerhalb des Dokumentes

\usepackage{hyperref}

% Für Layout von Links

\hypersetup{
	citecolor=black,
	colorlinks=true,
	linkcolor=black,
	urlcolor=blue,
}

% Verschiedene Mathematik-Hilfen

\newcommand \e[1]{\cdot10^{#1}}
\newcommand\p{\partial}

\newcommand\half{\frac 12}
\newcommand\shalf{\sfrac12}

\newcommand\skp[2]{\left\langle#1,#2\right\rangle}
\newcommand\mw[1]{\left\langle#1\right\rangle}
\newcommand \eexp[1]{\mathrm{e}^{#1}}
\newcommand \dexp[1]{\exp\del{#1}}
\newcommand \dsin[1]{\sin\del{#1}}
\newcommand \dcos[1]{\cos\del{#1}}
\newcommand \dtan[1]{\tan\del{#1}}
\newcommand \darccos[1]{\arcos\del{#1}}
\newcommand \darcsin[1]{\arcsin\del{#1}}
\newcommand \darctan[1]{\arctan\del{#1}}

% Nabla und Kombinationen von Nabla

\renewcommand\div[1]{\skp{\nabla}{#1}}
\newcommand\rot{\nabla\times}
\newcommand\grad[1]{\nabla#1}
\newcommand\laplace{\triangle}
\newcommand\dalambert{\mathop{{}\Box}\nolimits}

%Für komplexe Zahlen

\newcommand \ii{\mathrm i}
\renewcommand{\Im}{\mathop{{}\mathrm{Im}}\nolimits}
\renewcommand{\Re}{\mathop{{}\mathrm{Re}}\nolimits}

%Für Bra-Ket-Notation

\newcommand\bra[1]{\left\langle#1\right|}
\newcommand\ket[1]{\left|#1\right\rangle}
\newcommand\braket[2]{\left\langle#1\left.\vphantom{#1 #2}\right|#2\right\rangle}
\newcommand\braopket[3]{\left\langle#1\left.\vphantom{#1 #2 #3}\right|#2\left.\vphantom{#1 #2 #3}\right|#3\right\rangle}


\newcounter{thezettel}
\setcounter{thezettel}{8}
\renewcommand\thesection{\arabic{thezettel}.\arabic{section}}

\newcommand{\ui}[1]{\int_{-\infty}^{\infty}\dif {#1}\;}
\newcommand\ccancel[2][black]{\renewcommand\CancelColor{\color{#1}}\cancel{#2}}


\title{physik421 - Übung \arabic{thezettel}}
\author{Lino Lemmer \\ \small{l2@uni-bonn.de} \and Frederike Schrödel \and Simon Schlepphorst\\ \small{s2@uni-bonn.de}}

\begin{document}
\maketitle

\section{}

\section{}

\section{Heisenbergdarstellung von Operatoren}

Gegeben:
\begin{align*}
 \hat H = \frac{\hat p^2}{2m} - qE\hat x
\end{align*}

Zu zeigen:
\begin{align}
 \sbr{\hat A^n, \hat B} = n \hat A^{n-1} \sbr{\hat A, \hat B} \label{8.3-Induktion}
\end{align}
Beweis durch vollständige Induktion:
\begin{align*}
\intertext{$n = 1$}
 \sbr{\hat A, \hat B} &= 1 \hat A^0 \sbr{\hat A, \hat B}
 \intertext{Angenommen es gelte (\ref{8.3-Induktion}), Intduktionsschritt mit $n$}
 \sbr{\hat A^{n+1}, \hat B} &= \hat A^{n+1} \hat B - \hat B \hat A^{n+1}\\
 &= \hat A^{n} \hat A \hat B - \hat A^{n} \hat B \hat A + \hat A^{n} \hat B \hat A - \hat B \hat A^{n} \hat A\\
 &= \hat A^{n} \sbr{\hat A, \hat B} + \sbr{\hat A^n, \hat B} \hat A\\
 &= \hat A^{n} \sbr{\hat A, \hat B} + n \hat A^n \sbr{\hat A. \hat B}\\
 &= \del{n+1} \hat A^n \sbr{\hat A, \hat B}\\
 \\
 \implies \sbr{\hat A^n, \hat B} &= n\hat A^{n-1} \sbr{\hat A, \hat B}
\end{align*}

\begin{align*}
 \hat{\vec H} &= \hat H_x + \hat H_y + \hat H_z\\
 &\overset{?}{=} \frac{\del{\hat p_x + \hat p_y + \hat p_z}^2}{2m} - qE \vec e_x\\
 &=
\end{align*}


\begin{align*}
 \hat p_H (t) &= \ee^{\ii \hat H t /\hbar} \hat p \ee^{-\ii \hat H t / \hbar}\\
 &= \sum_{n=0}^\infty \frac{\del{\ii \hat H t /\hbar}^n}{n!} \hat p \sum_{n=0}^\infty \frac{\del{-\ii \hat H t / \hbar}^n}{n!}\\
 &= 
\end{align*}



\section{Zeitentwicklung der Matrix-Darstellung von Operatoren im Wechselwirkungsbild}

Gegeben ist ein Operator $\vec{q}_I$ im Wechselwirkungsbild. Es soll gezeigt werden, dass die Zeitableitung gegeben ist durch:
\[
	\ii\hbar\dod{}{t}\vec{q}_I=\ii\hbar\dpd{}{t}\vec{q}_I+\left[\vec{H}_0,\vec{q}_I\right]
\]

Im Wechselwirkungsbild benutzt man den Hamilton-Operator $H=H_0+H_1$, wobei $H_0$ das Wechelwirkungsfreihe System und $H_1$ die Wechselwirkung beschreiben. Dabei gilt für die Wellenfunktion $\Psi_I$ im Wechselwirkungsbild (Nolting, Seite 221):
\begin{align*}
	\ket{\Psi_I\del{t}}=U_0\del{t_0,t}\ket{\Psi\del{t}}&&\text{mit }U_0\del{t_0,t}=\ee^{\frac{\ii}{\hbar}\vec{H}_0\del{t-t_0}}
\end{align*}
Soll nun eine Observable A in das Wechselbild transformiert werden folgt:
\begin{align*}
	\braopket{\Psi}{A}{\Psi}&=\braopket{\Psi}{U_0\del{t_0,t}U_0^{-1}\del{t_0,t}AU_0\del{t_0,t}U_0^{-1}\del{t_0,t}}{\Psi}\\
	&=\braopket{\Psi_I}{U_0^{-1}\del{t_0,t}AU_0\del{t_0,t}}{\Psi_I}\\
\end{align*}
Der transformierte Operator $A_I$ ist also $U_0^{-1}\del{t_0,t}AU_0\del{t_0,t}$. Damit lässt sich nun die gewünschte Geleichung herleiten (zur Übersichtlichkeit lasse ich die $\del{t_0,t}$-Abhängigkeit weg:
\begin{align*}
	\ii\hbar\dod{}{t}\vec{q}_I&=\ii\hbar\dod{}{t}\del{U_0^{-1}\vec{q}U_0}\\
	&=\ii\hbar\dpd{U_0^{-1}}{t}\vec{q}U_0+\ii\hbar U_0^{-1}\dpd{\vec{q}}{t}U_0+\ii\hbar U_0^{-1}\vec{q}\dpd{U_0}{t}\\
	\intertext{%
		mit $\dpd{U_0}{t}=\frac{\ii}{\hbar}\vec{H}_0U_0$ und $\dpd{U_0^{-1}}{t}=-\frac{\ii}{\hbar}U_0^{-1}\vec{H}_0$
	}
	&=\vec{H}_0U_0^{-1}\vec{q}U_0+\ii\hbar U_0^{-1}\dpd{\vec{q}}{t}U_0-U_0^{-1}\vec{q}U_0\vec{H}_0\\
	\intertext{%
		mit $\vec{q}_I=U_0\vec{q}U_0^{-1}$ und $\dpd{\vec{q}_I}{t}=U_0\dpd{\vec{q}}{t}U_0^{-1}$
	}
	&=\vec{H}_0\vec{q}_I-\vec{q}_I\vec{H}_0+\ii\hbar\dpd{\vec{q}_I}{t}\\
	&=\left[\vec{H}_0,\vec{q}_I\right]+\ii\hbar\dpd{\vec{q}_I}{t}
\end{align*}


\end{document}
