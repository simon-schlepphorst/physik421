% Für Seitenformatierung

\documentclass[DIV=15]{scrartcl}

% Zeilenumbrüche

\parindent 0pt
\parskip 6pt

% Für deutsche Buchstaben und Synthax

\usepackage[ngerman]{babel}

% Für Auflistung mit speziellen Aufzählungszeichen

\usepackage{paralist}

% zB für \del, \dif und andere Mathebefehle

\usepackage{amsmath}
\usepackage{commath}
\usepackage{amssymb}

% Für Literatur/bibliography

\usepackage[backend=biber , style=alphabetic , hyperref=true]{biblatex}

% Für \SIunit[]{} und \num in deutschem Stil

\usepackage[output-decimal-marker={,}]{siunitx}
\DeclareSIUnit\clight{\ensuremath{c}}

% Schriftart und encoding

\usepackage[utf8]{inputenc}
% Bitstream charter als default
\usepackage[charter, greekuppercase=italicized]{mathdesign}
% Lato, als sans default
\renewcommand{\sfdefault}{fla}

% Für \sfrac{}{}, also inline-frac

\usepackage{xfrac}

% Für Einbinden von pdf-Grafiken

\usepackage{graphicx}

% Umfließen von Bildern

\usepackage{floatflt}

% Für weitere Farben

\usepackage{color}

% Für Streichen von z.B. $\rightarrow$

\usepackage{centernot}

% Für Befehl \cancel{}

\usepackage{cancel}

% Für Links nach außen und innerhalb des Dokumentes

\usepackage{hyperref}

% Für Layout von Links

\hypersetup{
	citecolor=black,
	colorlinks=true,
	linkcolor=black,
	urlcolor=blue,
}

% Verschiedene Mathematik-Hilfen

\newcommand \e[1]{\cdot10^{#1}}
\newcommand\p{\partial}

\newcommand\half{\frac 12}
\newcommand\shalf{\sfrac12}

\newcommand\skp[2]{\left\langle#1,#2\right\rangle}
\newcommand\mw[1]{\left\langle#1\right\rangle}
\newcommand \eexp[1]{\mathrm{e}^{#1}}
\newcommand \dexp[1]{\exp\del{#1}}
\newcommand \dsin[1]{\sin\del{#1}}
\newcommand \dcos[1]{\cos\del{#1}}
\newcommand \dtan[1]{\tan\del{#1}}
\newcommand \darccos[1]{\arcos\del{#1}}
\newcommand \darcsin[1]{\arcsin\del{#1}}
\newcommand \darctan[1]{\arctan\del{#1}}

% Nabla und Kombinationen von Nabla

\renewcommand\div[1]{\skp{\nabla}{#1}}
\newcommand\rot{\nabla\times}
\newcommand\grad[1]{\nabla#1}
\newcommand\laplace{\triangle}
\newcommand\dalambert{\mathop{{}\Box}\nolimits}

%Für komplexe Zahlen

\newcommand \ii{\mathrm i}
\renewcommand{\Im}{\mathop{{}\mathrm{Im}}\nolimits}
\renewcommand{\Re}{\mathop{{}\mathrm{Re}}\nolimits}

%Für Bra-Ket-Notation

\newcommand\bra[1]{\left\langle#1\right|}
\newcommand\ket[1]{\left|#1\right\rangle}
\newcommand\braket[2]{\left\langle#1\left.\vphantom{#1 #2}\right|#2\right\rangle}
\newcommand\braopket[3]{\left\langle#1\left.\vphantom{#1 #2 #3}\right|#2\left.\vphantom{#1 #2 #3}\right|#3\right\rangle}


\newcounter{thezettel}
\setcounter{thezettel}{11}
\renewcommand\thesection{\arabic{thezettel}.\arabic{section}}



\title{physik421 - Übung \arabic{thezettel}}
\author{Lino Lemmer \\ \small{l2@uni-bonn.de} \and Frederike Schrödel \and Simon Schlepphorst\\ \small{s2@uni-bonn.de}}

\begin{document}
\maketitle

\section{Radialimpuls} % (fold)
\label{sec:Radialimpuls}
 
% section Radialimpuls (end)

\section{Kugelflächenfunktion in kartesischen Koordinaten} % (fold)
\label{sec:Kugelflächenfunktion in kartesischen Koordinaten}

Gesucht ist $r^lY_{l\,m}$ in Kugel- und kartesischen Koordinaten für $l=0,1$ und $-l\leq m \leq +l$. Gegeben ist die allgemeine Form der Kugelflächenfunktionen
\begin{align*}
    Y_{l\,m}(\theta,\phi) &= \frac1{\sqrt{2\piup}} N_{l\,m} P_{l\,m}(\cos\theta)\eexp{\ii m\phi},
    \intertext{mit}
    N_{l\,m} &= \sqrt{\frac{\del{l+\half}\del{l-m}!}{\del{l+m}!}}
    \intertext{und}
    P_{l\,m}(x) &= \frac{(-1)^m}{2^ll!}\del{1-x^2}^{\frac m2} \dod[l+m]{}{x}\del{x^2-1}^l.
    \intertext{%
        Wir beginnen mit $l=0$ und $m=0$:
    }
    N_{0\,0} &= \sqrt{\half} \\
    P_{0\,0} &= 1 \\
    \implies\qquad r^0Y_{0\,0} &= \frac{1}{2\sqrt\piup}.
    \intertext{%
        Diese Funktion ist unabhängig von den Koordinaten.\newline
        Nun schauen wir uns $l=1$ an und beginnen dabei mit $m=-1$:
    }
    N_{1\,-1} &= \sqrt{3} \\
    P_{1\,-1}(x) &= -\half\frac{1}{\sqrt{1-x^2}}\del{x^2-1} \\
    P_{1\,-1}(\cos\theta) &= -\half\frac1{\sin\theta}\del{-\sin^2\theta} \\
                          &= \half\sin\theta \\
    \implies \qquad rY_{1\,-1} &= \sqrt{\frac{3}{8\piup}} r \sin\theta\eexp{-\ii\phi} \\
                               &= \sqrt{\frac{3}{8\piup}} \del{r \sin\theta\cos\phi - \ii\sin\theta\sin\phi} \\
                               &= \sqrt{\frac{3}{8\piup}} \del{x-\ii y}.
    \intertext{%
        Nun $m=0$:
    }
    N_{1\,0} &= \sqrt{\frac32} \\
    P_{1\,0}(x) &= \half \dod{}{x} \del{x^2-1} = x \\
    P_{1\,0}(\cos\theta) &= \cos\theta \\
    \implies \qquad rY_{1\,0} &= \sqrt{\frac3{4\piup}} r\cos\theta \\
                              &= \sqrt{\frac3{4\piup}} z.
    \intertext{%
        Und zu guter letzt $m=1$:
    }
    N_{1\,1} &= \sqrt{\frac34} \\
    P_{1\,1}(x) &= -\half \sqrt{1-x^2} \dpd[2]{}{x}\del{x^2-1} = -\sqrt{1-x^2} \\
    P_{1\,1}(\cos\theta) &= -\sin\theta \\
    \implies \qquad rY_{1\,1} &= - \sqrt{\frac3{8\piup}} r\sin\theta\eexp{\ii\phi} \\
                              &= - \sqrt{\frac3{8\piup}} \del{r\sin\theta\cos\phi + \ii r\sin\theta\sin\phi} \\
                              &= - \sqrt{\frac3{8\piup}} \del{x + \ii y}
\end{align*}

% section Kugelflächenfunktion in kartesischen Koordinaten (end)

\section{Zentralpotenzial mit Korrektur} % (fold)
\label{sec:Zentralpotenzial mit Korrektur}

% section Zentralpotenzial mit Korrektur (end)

\section{Elektron im Wasserstoffatom} % (fold)
\label{sec:Elektron im Wasserstoffatom}

Gegeben ist die Wellenfunktion
\begin{align*}
    \Psi(r,\theta,\phi) &= Ar\dexp{-\frac{r}{2a_0}}Y_{1\,1}(\theta,\phi),
    \intertext{%
        mit
    }
    a_0 &= \frac{4\piup\epsilon_0\hbar^2}{m_ee^2}.
    \intertext{%
        Zu zeigen ist nun, dass diese Wellenfunktion eine Eigenfunktion des Hamiltonoperators
    }
    H &= -\frac{\hbar^2}{2m_e}\laplace - \frac{e^2}{4\piup\epsilon_0r}
    \intertext{%
        ist. Dafür benötigen wir zunächst den Laplace-Operator in Kugelkoordinaten:
    }
    \laplace\Psi &= \frac1{r^2} \dpd{}{r} \del{r^2\dpd{\Psi}{r}} + \frac1{r^2\sin\theta} \dpd{}{\theta} \del{\sin\theta \dpd\Psi\theta } + \frac1{r^2\sin^2\theta}\dpd[2]\Psi\phi.
    \intertext{%
        Um dies zu berechnen nehmen wir uns die einzelnen Komponenten vor. Wir beginnen mit dem Radialteil:
    }
    \frac1{r^2} \dpd{}{r} \del{r^2\dpd{\Psi}{r}} &= Y_{1\,1}\frac1{r^2}\dpd{}{r}\del{A\dexp{-\frac{r}{2a_0}}\del{ r^2 - \frac{r^3}{2a_0}}} \\
                                                 &= Y_{1\,1}\frac1{r^2}\del{A\dexp{-\frac{r}{2a_0}}\del{-\frac{r^2}{2a_0} + \frac{r^3}{4a_0^2} + 2r - \frac{3r^2}{2a_0}}} \\
                                                 &= Y_{1\,1}\dexp{-\frac{r}{2a_0}}\del{-\frac{2}{a_0} + \frac{r^3}{4a_0^2} + \frac2r} \\
                                                 &= -\sqrt{\frac3{8\piup}}\sin\theta\eexp{\ii\phi} Y_{1\,1}\dexp{-\frac{r}{2a_0}}\del{-\frac{2}{a_0} + \frac{r^3}{4a_0^2} + \frac2r}. \label{eq:radial} \mantag
    \intertext{%
        Als Nächstes schauen wir uns den $\theta$-Teil an:
    }
    \frac1{r^2\sin\theta} \dpd{}{\theta} \del{\sin\theta \dpd\Psi\theta } &= \frac{A}{r\sin\theta}\dexp{-\frac r{2a_0}}\dpd{}{\theta}\del{\sin\dpd{Y_{1\,1}}{\theta}} \\
    &= -\sqrt{\frac3{8\piup}}\eexp{\ii\phi} \frac{A}{r\sin\theta}\dexp{-\frac r{2a_0}}\dpd{}{\theta}\del{\sin\dpd{\sin\theta}{\theta}} \\
                                                                          &= -\sqrt{\frac3{8\piup}}\eexp{\ii\phi} \frac{A}{r\sin\theta}\dexp{-\frac r{2a_0}}\dpd{}{\theta}\sin\theta\cos\theta \\
                                                                          &= -\sqrt{\frac3{8\piup}}\eexp{\ii\phi} \frac{A}{r\sin\theta}\dexp{-\frac r{2a_0}}\del{\cos^2\theta-\sin^2\theta}. \label{eq:theta}\mantag
    \intertext{%
        Zum Schluss der $\phi$-Teil:
    }
    \frac1{r^2\sin^2\theta}\dpd[2]\Psi\phi &= \frac A{r\sin^2\theta}\dexp{-\frac r{2a_0}}\dpd[2]{Y_{1\,1}}{\phi} \\
                                           &= -\sqrt{\frac3{8\piup}} \frac{A}{r\sin\theta}\dexp{-\frac r{2a_0}} \dpd[2]{\eexp{\ii\phi}}
    {\phi} \\
    &= \sqrt{\frac3{8\piup}} \frac{A}{r\sin\theta}\dexp{-\frac r{2a_0}} \eexp{\ii\phi}. \label{eq:phi}\mantag
    \intertext{%
        Mit \eqref{eq:radial}, \eqref{eq:theta} und \eqref{eq:phi} ergibt sich nun
    }
    \laplace\Psi &= -\sqrt{\frac3{8\piup}}\sin\theta\eexp{\ii\phi} Y_{1\,1}\dexp{-\frac{r}{2a_0}}\del{-\frac{2}{a_0} + \frac{r^3}{4a_0^2} + \frac2r} \\
                 &\phantom{=} -\sqrt{\frac3{8\piup}}\eexp{\ii\phi} \frac{A}{r\sin\theta}\dexp{-\frac r{2a_0}}\del{\cos^2\theta-\sin^2\theta} \\
                 &\phantom{=} -\sqrt{\frac3{8\piup}}\eexp{\ii\phi} \frac{A}{r\sin\theta}\dexp{-\frac r{2a_0}}\del{\cos^2\theta-\sin^2\theta}.
    \intertext{%
        Ein ersten Ausklammern führt uns auf
    }
    &= -\sqrt{\frac3{8\piup}} \eexp{\ii\phi} A\dexp{-\frac r{2a_0}} \del{-\frac{2\sin\theta}{a_0} + \frac{2\sin\theta}r + \frac{r\sin\theta}{4a_0^2} + \frac{\cos^2\theta}{r\sin\theta} - \frac{\sin\theta}r - \frac1{r\sin\theta}} \\
    &= -\sqrt{\frac3{8\piup}} \eexp{\ii\phi} A\dexp{-\frac r{2a_0}} \del{ -\frac{2\sin\theta}{a_0} + \frac{r\sin\theta}{4a_0^2} + \frac{\sin\theta}r + \frac{\cos^2\theta - 1}{r\sin\theta}} \\
    &= -\sqrt{\frac3{8\piup}} \eexp{\ii\phi} A\dexp{-\frac r{2a_0}} \del{ -\frac{2\sin\theta}{a_0} + \frac{r\sin\theta}{4a_0^2} + \cancel{\frac{\sin\theta}r} - \cancel{\frac{\sin^2\theta}{r\sin\theta}}} \\
    &= -\sqrt{\frac3{8\piup}} \sin\theta \eexp{\ii\phi} A r\dexp{-\frac r{2a_0}} \del{\frac1{4a_0^2} - \frac2{a_0r}} \\
    &= \del{\frac1{4a_0^2} - \frac2{a_0r}} \Psi.
    \intertext{%
        Nun können wir den Hamiltonoperator auf die Wellenfunktion anwenden:
    }
    H\Psi &= \del{-\frac{\hbar^2}{2m_e}\laplace - \frac{e^2}{4\piup\epsilon_0r}}\Psi \\
          &= \del{-\frac{\hbar^2}{2m_e}\del{\frac1{4a_0^2} - \frac2{a_0r}} - \frac{e^2}{4\piup\epsilon_0r}}\Psi .
    \intertext{%
        Wir setzen $a_0$ ein und erhalten
    }
    &= \del{-\frac{\hbar^2}{2m_e}\frac{m_e^2e^4}{64\piup^2\epsilon_0^2\hbar^4} + \cancel{\frac{\hbar^2}{2m_e}\frac{2m_ee^2}{4\piup\epsilon_0\hbar^2r}} - \cancel{\frac{e^2}{4\piup\epsilon_0r}}}\Psi \\
    &= -\frac{m_ee^4}{128\piup^2\epsilon_0^2\hbar^2}\Psi.
\end{align*}
$\Psi$ ist also eine Eigenfunktion vom Hamiltonoperator $H$, mit dem Energieeigenwert
$-\frac{m_ee^4}{128\piup^2\epsilon_0^2\hbar^2}$. Dies kommt von den Einheiten her auch hin.

Der Zustand des Elektrons ist offensichtlich durch die Quantenzahlen $l$ und $m$ gekennzeichnet.
% section Elektron im Wasserstoffatom (end)

\end{document}
