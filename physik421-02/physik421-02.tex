% Für Seitenformatierung

\documentclass[DIV=15]{scrartcl}

% Zeilenumbrüche

\parindent 0pt
\parskip 6pt

% Für deutsche Buchstaben und Synthax

\usepackage[ngerman]{babel}

% Für Auflistung mit speziellen Aufzählungszeichen

\usepackage{paralist}

% zB für \del, \dif und andere Mathebefehle

\usepackage{amsmath}
\usepackage{commath}
\usepackage{amssymb}

% Für Literatur/bibliography

\usepackage[backend=biber , style=alphabetic , hyperref=true]{biblatex}

% Für \SIunit[]{} und \num in deutschem Stil

\usepackage[output-decimal-marker={,}]{siunitx}
\DeclareSIUnit\clight{\ensuremath{c}}

% Schriftart und encoding

\usepackage[utf8]{inputenc}
% Bitstream charter als default
\usepackage[charter, greekuppercase=italicized]{mathdesign}
% Lato, als sans default
\renewcommand{\sfdefault}{fla}

% Für \sfrac{}{}, also inline-frac

\usepackage{xfrac}

% Für Einbinden von pdf-Grafiken

\usepackage{graphicx}

% Umfließen von Bildern

\usepackage{floatflt}

% Für weitere Farben

\usepackage{color}

% Für Streichen von z.B. $\rightarrow$

\usepackage{centernot}

% Für Befehl \cancel{}

\usepackage{cancel}

% Für Links nach außen und innerhalb des Dokumentes

\usepackage{hyperref}

% Für Layout von Links

\hypersetup{
	citecolor=black,
	colorlinks=true,
	linkcolor=black,
	urlcolor=blue,
}

% Verschiedene Mathematik-Hilfen

\newcommand \e[1]{\cdot10^{#1}}
\newcommand\p{\partial}

\newcommand\half{\frac 12}
\newcommand\shalf{\sfrac12}

\newcommand\skp[2]{\left\langle#1,#2\right\rangle}
\newcommand\mw[1]{\left\langle#1\right\rangle}
\newcommand \eexp[1]{\mathrm{e}^{#1}}
\newcommand \dexp[1]{\exp\del{#1}}
\newcommand \dsin[1]{\sin\del{#1}}
\newcommand \dcos[1]{\cos\del{#1}}
\newcommand \dtan[1]{\tan\del{#1}}
\newcommand \darccos[1]{\arcos\del{#1}}
\newcommand \darcsin[1]{\arcsin\del{#1}}
\newcommand \darctan[1]{\arctan\del{#1}}

% Nabla und Kombinationen von Nabla

\renewcommand\div[1]{\skp{\nabla}{#1}}
\newcommand\rot{\nabla\times}
\newcommand\grad[1]{\nabla#1}
\newcommand\laplace{\triangle}
\newcommand\dalambert{\mathop{{}\Box}\nolimits}

%Für komplexe Zahlen

\newcommand \ii{\mathrm i}
\renewcommand{\Im}{\mathop{{}\mathrm{Im}}\nolimits}
\renewcommand{\Re}{\mathop{{}\mathrm{Re}}\nolimits}

%Für Bra-Ket-Notation

\newcommand\bra[1]{\left\langle#1\right|}
\newcommand\ket[1]{\left|#1\right\rangle}
\newcommand\braket[2]{\left\langle#1\left.\vphantom{#1 #2}\right|#2\right\rangle}
\newcommand\braopket[3]{\left\langle#1\left.\vphantom{#1 #2 #3}\right|#2\left.\vphantom{#1 #2 #3}\right|#3\right\rangle}


\newcounter{thezettel}
\setcounter{thezettel}{2}
\renewcommand\thesection{\arabic{thezettel}.\arabic{section}}


\title{physik421 - Übung \arabic{thezettel}}
\author{Lino Lemmer \\ \small{l2@uni-bonn.de} \and Frederike Schrödel \and Simon Schlepphorst\\ \small{s2@uni-bonn.de}}

\begin{document}
\maketitle

\section{Entartung von Energieniveaus}

Zu zeigen ist, dass bei einer eindimensionalen, stationären Schrödingergleichung mit einem zeitunabhängigen Potential $V(x)$ kein Energieniveau des diskreten Spektrums entartet ist.
\[
    \left(-\frac{\hbar^2}{2m}\dpd[2]{}{x}+V(x)\right)\psi(x,t) = E\psi(x,t)
\]
Für zwei Wellenfunktionen gilt also:

\begin{align*}
    H\psi_1&=E\psi_1	&	H\psi_2&=E\psi_2		\\
    E\psi_1&=\left(-\frac{\hbar^2}{2m}\dpd[2]{}{x}+V(x)\right)\psi_1	&
    E\psi_2&=\left(-\frac{\hbar^2}{2m}\dpd[2]{}{x}+V(x)\right)\psi_2	\\
    E\psi_1\psi_2&=\left(\left(-\frac{\hbar^2}{2m}\dpd[2]{}{x} +
    V(x)\right)\psi_1\right)\psi_2
    &
    E\psi_2\psi_1&=\left(\left(-\frac{\hbar^2}{2m}\dpd[2]{}{x} +
    V(x)\right)\psi_2\right)\psi_1
\end{align*}

Beide Gleichungen werden nun gleich gesetzt und die Klammern aufgelöst:
\begin{align*}
    \left(\left(-\frac{\hbar^2}{2m}\dpd[2]{}{x} + V(x)\right)\psi_1\right)\psi_2 &= \left(\left(-\frac{\hbar^2}{2m}\dpd[2]{}{x} + V(x)\right)\psi_2\right)\psi_1	\\
  \cancel{-\frac{\hbar^2}{2m}}\psi''_1\psi_2+\cancel{V(x)\psi_1\psi_2} &= \cancel{-\frac{\hbar^2}{2m}}\psi_1\psi''_2+\cancel{V(x)\psi_1\psi_2}
    \intertext{%
        Das liefert also:
    }
    \psi''_1\psi_2 - \psi_1\psi''_2 &= 0
  \intertext{%
      Nun Integrieren wir zwei mal partiell:
  }
  \int\!\dif x \, \psi_1''\psi_2 - \psi_1\psi''_2 &= \psi'_1\psi_2 - \cancel{\int \!\dif x \, \psi'_1\psi'_2} - \psi_1\psi'_2 + \cancel{\int \!\dif x \,\psi'_1\psi'_2} + C \\
  &= \psi'_1\psi_2 - \psi_1\psi'_2
\end{align*}


\section{$\delta$-Potenzial}

In dieser Aufgabe sollen Eigenwerte und normierte Eigenfunktionen eines $\delta$-Potentials $V(x)=-\alpha\delta(x)$ berechnet werden. Dabei sollen die Energien als negativ betrachtet werden. Es ergibt sich also die folgende Schrödingergleichung:
\begin{align*}
  \left(-\frac{\hbar^2}{2m}\dpd[2]{}{x}-\alpha\delta(x)\right)\psi(x,t)&=-E\psi(x,t)	\\
  -\frac{\hbar^2}{2m}\psi''-\alpha\delta(x)\psi&=-E\psi
    \intertext{%
        Für die Grenzbedingung bei $x=0$:
    }
  E \int^\epsilon_{-\epsilon}\!\dif x\, \psi &= \int^\epsilon_{-\epsilon}\!\dif x\, \frac{\hbar^2}{2m} \psi'' + \int^\epsilon_{-\epsilon}\!\dif x\, \alpha\delta(x)\psi		\\
  &= \frac{\hbar^2}{2m}\left(\psi'(\epsilon)-\psi'(-\epsilon)\right)+\alpha\psi(0)
    \intertext{%
        Nun wird der Grenzwert $\epsilon\rightarrow 0$ gebildet:
    }
  \lim\limits_{\epsilon \to 0} \left(\frac{\hbar^2}{2m}\left(\psi'(\epsilon)-\psi'(-\epsilon)\right)+\alpha\psi(0)\right)	\\
  \implies \psi'(0^+)-\psi'(0^-)&=-\frac{2m\alpha}{\hbar}\psi(0)
\end{align*}
Damit die Grenzwerte $0^+$ und $0^-$ existieren müssen die Wellengleichung $\psi$ und ihre Ableitung an der Stelle $x=0$ stetig sein.

\section{Stückweise konstantes Potenzial}

\subsection{Anschlussbedingungen}

Die Schrödinger-Gleichung lautet
\begin{align*}
    \ii\hbar\dpd{}{t} \psi(q,t) &= \del{-\frac{\hbar^2}{2m}\dpd[2]{}{q} + V(q)} \psi(q,t),
    \intertext{%
        mit
    }
    V(q) &=
    \begin{cases}
        V_1 & ,-\infty<q<-q_0 \text{ (Bereich 1)}\\
        0 & ,-q_0<q<+q_0 \text{ (Bereich 2)} \\
        V_3 & ,+q_0<q<+\infty \text{ (Bereich 3)}
    \end{cases}.
    \intertext{%
        Jetzt machen wir den Faktorisierungsansatz $\psi(q,t) = u(q)v(t)$:
    }
    \frac{\ii\hbar\dpd{v}{t}}{v(t)} &= \frac{-\frac{\hbar^2}{2m}\dpd[2]{u}{q}}{u(q)} + V(q) \overset{!}{=} E,
    \intertext{%
        wobei $E$ eine konstante ist. Für die zeitabhängige Lösung ergibt sich
    }
    v(t) &= \eexp{-\ii\frac{E}{\hbar}t}.
    \intertext{%
        Nun betrachten wir die ortsabhängige Funktion.
    }
    \frac{\hbar^2}{2m} \dpd[2]{u}{q} &= \del{V(q) - E} u(q) \\
    \iff \qquad \dpd[2]{u}{q} &= \frac{2m}{\hbar^2}\del{V(q) - E}u
    \intertext{%
        Für Bereich 1 ergibt sich mit $\kappa_1^2=\frac{2m}{\hbar^2}\del{V_1 -E}$ die Differentialgleichung
    }
    \dpd[2]{u_1}{q} &= \kappa_1^2u_1,
    \intertext{%
        mit der Lösung
    }
    u_1(q) &= a_1\eexp{\kappa_1 q} + a_2\eexp{-\kappa_1 q}.
    \intertext{%
        Bereich 3 lässt sich mit $\kappa_3^2=\frac{2m}{\hbar^2}\del{V_3 -E}$ analog lösen. Wir erhalten
    }
    u_3(q) &= c_1\eexp{\kappa_3 q} + c_2\eexp{-\kappa_3 q}.
    \intertext{%
        Für Bereich 2 erhalten wir mit $k^2=\frac{2m}{\hbar^2}E$ die Differentialgleichung
    }
    \dpd[2]{u_2}{q} &= -k^2u_2,
    \intertext{%
        mit der Lösung
    }
    u_2(q) &= b_1\eexp{\ii kq} + b_2\eexp{-\ii kq}.
    \intertext{%
        Da sowohl die Funktion, als auch die Ableitung stetig sein muss erhalten wir die Randbedingungen
    }
    u_1\del{-q_0} &= u_2\del{-q_0} \\
    u_1'\del{-q_0} &= u_2'\del{-q_0} \\
    u_2\del{q_0} &= u_3\del{q_0} \\
    u_2'\del{q_0} &= u_3'\del{q_0},
    \intertext{%
        oder ausgeschrieben
    }
    a_1\eexp{-\kappa_1 q_0} + a_2 \eexp{\kappa_1 q_0} &= b_1 \eexp{-\ii kq_0} + b_2 \eexp{\ii kq_0} \\
    \kappa_1\del{a_1\eexp{-\kappa_1 q_0} - a_2\eexp{\kappa_1 q_0}} &= \ii k\del{ b_1 \eexp{-\ii kq_0} - b_2 \eexp{\ii kq_0}} \\
    b_1 \eexp{\ii kq_0} + b_2 \eexp{-\ii kq_0} &= c_1\eexp{\kappa_3 q_0} + c_2 \eexp{-\kappa_3 q_0} \\
    \ii k\del{ b_1 \eexp{\ii kq_0} - b_2 \eexp{-\ii kq_0}} &= \kappa_3\del{c_1\eexp{\kappa_3 q_0} - c_2\eexp{-\kappa_3 q_0}} \\
\end{align*}

\subsection{Eigenenergien und transzendente Gleichung}

In den Randbedingungen muss der Faktor $a_2$ verschwinden, damit die
Wellenfunktion normierbar ist, ebenso der Faktor $c_1$. Wenn wir die zweite
Randbedingung durch $\kappa_1$ teilen und von der ersten subtrahieren erhalten
wir
\begin{align*}
    && 0 &= \frac{\ii k - \kappa_1}{\kappa_1}b_1\eexp{-\ii kq_0} - \frac{\ii k + \kappa_1}{\kappa_1}b_2\eexp{\ii kq_0}.
    \intertext{%
        Nach einem kleinen Bisschen Umformen erhalten wir daraus
    }
    && \frac{b_1}{b_2} &= \frac{\ii k + \kappa_1}{\ii k - \kappa_1}\eexp{2\ii kq_0}.
    \intertext{%
        Teilen wir die vierte Randbedingung durch $\kappa_3$ und addieren das Ergebnis zur Dritten, ergibt sich
    }
    && 0 &= \frac{\ii k + \kappa_3}{\kappa_3}b_1\eexp{\ii kq_0} - \frac{\ii k - \kappa_3}{\kappa_3}b_2\eexp{-\ii kq_0},
    \intertext{%
        aus dem man durch Umstellen
    }
    && \frac{b_1}{b_2} &= \frac{\ii k - \kappa_3}{\ii k + \kappa_3} \eexp{-2\ii kq_0}
    \intertext{%
        erhält. Nun setzen wir beide Ausdrücke für $\frac{b_1}{b_2}$ gleich:
    }
    && \frac{\ii k - \kappa_3}{\ii k + \kappa_3} \eexp{-2\ii kq_0} &= \frac{\ii k + \kappa_1}{\ii k - \kappa_1}\eexp{2\ii kq_0} \\
    &\iff & 1 &= \eexp{-4\ii kq_0}\frac{\del{\ii k - \kappa_1}\del{\ii k - \kappa_3}}{\del{\ii k + \kappa_3}\del{\ii k + \kappa_1}}
    \intertext{%
        Erweitern wir nun mit den komplex Konjugierten der ersten Klammer im Zähler und im Nenner, erhalten wir
    }
    && 1 &= \eexp{-4\ii kq_0}\frac{\del{\ii k - \kappa_1}\del{\ii k + \kappa_1}\del{\ii k - \kappa_3}^2}{\del{\ii k + \kappa_3}\del{\ii k - \kappa_3}\del{\ii k + \kappa_1}^2}.
    \intertext{%
        Die Multiplikation einer komplexen Zahl mit ihrer komplex Konjugierten ergibt ihr Betragsquadrat:
    }
    && 1 &= \eexp{-4\ii kq_0}\frac{\del{k^2 + \kappa_1^2}}{\del{k^2 + \kappa_3^2}}\frac{\del{\ii k - \kappa_3}^2}{\del{\ii k + \kappa_1}^2}
    \intertext{%
        Da nun gilt $V_i = \frac{\hbar^2\del{k^2+\kappa_i^2}}{2m}$, erhalten wir die Beziehung
    }
    && 1 &= \eexp{-4\ii kq_0}\frac{V_1}{V_3}\frac{\del{\ii k - \kappa_3}^2}{\del{\ii k + \kappa_1}^2}.
\end{align*}
Diese sieht der vorgegebenen transzendenten Gleichung immerhin ziemlich ähnlich.

\subsection{Bestimmungsgleichung}

\subsection{Diskretes Eigenwertspektrum}

\subsection{Eigenwerte bei unterschiedlichen Potenzialen}

\section{Streuung am Potenzialwall}

Stationäre Schrödingergleichung:
\begin{align*}
 \del{-\frac{\hbar^2}{2m} \dpd[2]{}{x} + V\del{x}} u\del{x} = E u\del{x}\\
 \intertext{mit}
 V\del{x} =
 \begin{cases}
  0 & \text{für } \abs{x} \geq \frac{a}{2}\\
  V_0 & \text{für } \abs{x} < \frac{a}{2}
 \end{cases}
\end{align*}


\subsection{Form der Wellenfunktion}
Für das von $t$ unabhängige Potenzial $V\del{x}$ wird die gegebene Welle $\del{0 < E < V_0}$ faktorisiert in $\psi = \ee^{\ii\del{kx-\omega t}} = v\del{t}u\del{x}$.
Des weiteren wird von dem Potenzial ein Teil reflektiert und ein Teil transmittiert.

$\abs{x} \geq \frac{a}{2}$:
\begin{align*}
 &-\frac{\hbar^2}{2m} \dpd[2]{u\del{x}}{x} = E u\del{x}\\
 \iff & \dpd[2]{u\del{x}}{x} = - \frac{2mE}{\hbar^2} u\del{x}\\
 \implies &u\del{x} = \ee^{\ii kx} \quad \text{mit} \quad k = \sqrt{\frac{2mE}{\hbar^2}}
\end{align*}
$\abs{x} < \frac{a}2$:
\begin{align*}
 &-\frac{\hbar^2}{2m} \dpd[2]{u\del{x}}{x} + V_0 = E u\del{x}\\
 \iff & \dpd[2]{u\del{x}}{x} = \frac{2m\del{V_0 - E}}{\hbar^2} u\del{x}\\
 \implies &u\del{x} = \ee^{qx} \quad \text{mit} \quad q = \sqrt{\frac{2m\del{V_0-E}}{\hbar^2}}
\end{align*}
\begin{align*}
 \implies \psi\del x = 
 \begin{cases}
  \ee^{\ii kx} + R\ee^{-\ii kx} & \text{für} \quad x \leq - \frac a2\\
  A\ee^{qx} + B\ee^{qx} & \text{für} \quad -\frac a2 < x < \frac a2\\
  T\ee^{\ii kx} & \text{für} \quad x \geq \frac a2
 \end{cases}
\end{align*}

\subsection{Anschlussbedingungen}

An den Übergangspunkten muss $\psi\del x$ stetig sein.
\begin{align*}
 \psi\del{-\frac a2}&: &\ee^{-\ii k a / 2} + R\ee^{\ii k a / 2} = A\ee^{-qa / 2} + B\ee^{qa / 2}\\
 \psi'\del{-\frac a2}&: &\ii k \ee^{-\ii k a / 2} - \ii k R \ee^{\ii a / 2} = qA\ee^{-qa / 2} - qB\ee^{qa / 2}\\
 \psi\del{\frac a2}&: & T\ee^{\ii ka/2} = A\ee^{q a/2} + B\ee^{-q a/2}\\
 \psi'\del{\frac a2}&: & \ii kT\ee^{\ii ka/2} = qA\ee^{q a/2} - q B\ee^{-q a/2}
\end{align*}

\begin{align*}
 \implies
 \begin{pmatrix}
  \del{1 - \ii \frac qk}\ee^{-qa/2} & \del{1 + \ii\frac qk}\ee^{qa/2}\\
  \del{1 + \ii \frac qk}\ee^{qa/2} & \del{1 - \ii\frac qk}\ee^{-qa/2}
 \end{pmatrix}
 \begin{pmatrix}
  A\\
  B
 \end{pmatrix}
 =
 \begin{pmatrix}
  2\ee^{-\ii k a /2}\\
  0
 \end{pmatrix}
\end{align*}
Definiere:
\begin{align*}
 &\alpha \equiv \del{1 - \ii \frac qk}\ee^{-qa/2} &&\beta \equiv \del{1 + \ii\frac qk}\ee^{qa/2} &&\gamma \equiv   2\ee^{-\ii k a /2}
\end{align*}
\begin{align*}
 \implies &A = \frac{\alpha \gamma}{\alpha^2 - \beta^2} &&B = -\frac{\beta \gamma}{\alpha^2 - \beta^2}
\end{align*}
Das noch auszurechnen wird unnötig hässlich.



\subsection{Reflexions- und Transmissionskoeffizient}

Aus den Anschlussbedingungen folgt:
\begin{align*}
 R = \frac{A\ee^{-qa/2} + B\ee^{qa/2} - \ee^{-\ii ka/2}}{\ee^{\ii ka/2}}\\
 T = \frac{A\ee^{qa/2} + B\ee^{-qa/2}}{\ee^{\ii ka/2}}
\end{align*}


\end{document}
