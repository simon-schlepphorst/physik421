% Für Seitenformatierung

\documentclass[DIV=15]{scrartcl}

% Zeilenumbrüche

\parindent 0pt
\parskip 6pt

% Für deutsche Buchstaben und Synthax

\usepackage[ngerman]{babel}

% Für Auflistung mit speziellen Aufzählungszeichen

\usepackage{paralist}

% zB für \del, \dif und andere Mathebefehle

\usepackage{amsmath}
\usepackage{commath}
\usepackage{amssymb}

% Für Literatur/bibliography

\usepackage[backend=biber , style=alphabetic , hyperref=true]{biblatex}

% Für \SIunit[]{} und \num in deutschem Stil

\usepackage[output-decimal-marker={,}]{siunitx}
\DeclareSIUnit\clight{\ensuremath{c}}

% Schriftart und encoding

\usepackage[utf8]{inputenc}
% Bitstream charter als default
\usepackage[charter, greekuppercase=italicized]{mathdesign}
% Lato, als sans default
\renewcommand{\sfdefault}{fla}

% Für \sfrac{}{}, also inline-frac

\usepackage{xfrac}

% Für Einbinden von pdf-Grafiken

\usepackage{graphicx}

% Umfließen von Bildern

\usepackage{floatflt}

% Für weitere Farben

\usepackage{color}

% Für Streichen von z.B. $\rightarrow$

\usepackage{centernot}

% Für Befehl \cancel{}

\usepackage{cancel}

% Für Links nach außen und innerhalb des Dokumentes

\usepackage{hyperref}

% Für Layout von Links

\hypersetup{
	citecolor=black,
	colorlinks=true,
	linkcolor=black,
	urlcolor=blue,
}

% Verschiedene Mathematik-Hilfen

\newcommand \e[1]{\cdot10^{#1}}
\newcommand\p{\partial}

\newcommand\half{\frac 12}
\newcommand\shalf{\sfrac12}

\newcommand\skp[2]{\left\langle#1,#2\right\rangle}
\newcommand\mw[1]{\left\langle#1\right\rangle}
\newcommand \eexp[1]{\mathrm{e}^{#1}}
\newcommand \dexp[1]{\exp\del{#1}}
\newcommand \dsin[1]{\sin\del{#1}}
\newcommand \dcos[1]{\cos\del{#1}}
\newcommand \dtan[1]{\tan\del{#1}}
\newcommand \darccos[1]{\arcos\del{#1}}
\newcommand \darcsin[1]{\arcsin\del{#1}}
\newcommand \darctan[1]{\arctan\del{#1}}

% Nabla und Kombinationen von Nabla

\renewcommand\div[1]{\skp{\nabla}{#1}}
\newcommand\rot{\nabla\times}
\newcommand\grad[1]{\nabla#1}
\newcommand\laplace{\triangle}
\newcommand\dalambert{\mathop{{}\Box}\nolimits}

%Für komplexe Zahlen

\newcommand \ii{\mathrm i}
\renewcommand{\Im}{\mathop{{}\mathrm{Im}}\nolimits}
\renewcommand{\Re}{\mathop{{}\mathrm{Re}}\nolimits}

%Für Bra-Ket-Notation

\newcommand\bra[1]{\left\langle#1\right|}
\newcommand\ket[1]{\left|#1\right\rangle}
\newcommand\braket[2]{\left\langle#1\left.\vphantom{#1 #2}\right|#2\right\rangle}
\newcommand\braopket[3]{\left\langle#1\left.\vphantom{#1 #2 #3}\right|#2\left.\vphantom{#1 #2 #3}\right|#3\right\rangle}


\newcounter{thezettel}
\setcounter{thezettel}{10}
\renewcommand\thesection{\arabic{thezettel}.\arabic{section}}



\title{physik421 - Übung \arabic{thezettel}}
\author{Lino Lemmer \\ \small{l2@uni-bonn.de} \and Frederike Schrödel \and Simon Schlepphorst\\ \small{s2@uni-bonn.de}}

\begin{document}
\maketitle

\section{Spinmatrizen} % (fold)
\label{sec:Spinmatrizen}

Gegeben sind die drei Pauli-Matrizen
\[
\sigma_x = \begin{pmatrix} 0 & 1 \\ 1 & 0 \end{pmatrix} \qquad
\sigma_y = \begin{pmatrix} 0 & -\ii \\ \ii & 0 \end{pmatrix} \qquad
\sigma_z = \begin{pmatrix} 1 & 0 \\ 0 & -1 \end{pmatrix}
\]
\subsection{Eigenschaften} % (fold)
\label{ssec:Eigenschaften}

\begin{align*}
    \sigma_x\sigma_y + \sigma_y\sigma_x &= 
    \begin{pmatrix}0 & 1 \\ 1 & 0 \end{pmatrix}\begin{pmatrix} 0 & -\ii \\ \ii & 0 \end{pmatrix} + \begin{pmatrix} 0 & -\ii \\ \ii & 0 \end{pmatrix}\begin{pmatrix}0 & 1 \\ 1 & 0 \end{pmatrix} \\
                     &= \begin{pmatrix} \ii & 0 \\ 0 & - \ii \end{pmatrix} + \begin{pmatrix} -\ii & 0 \\ 0 & \ii \end{pmatrix} \\
                     &= 0 \\
\sigma_x^2 &= \begin{pmatrix}0 & 1 \\ 1 & 0 \end{pmatrix}\begin{pmatrix}0 & 1 \\ 1 & 0 \end{pmatrix} \\
           &= \mathbb{1} \\
    \sigma_y^2 &=  \begin{pmatrix} 0 & -\ii \\ \ii & 0 \end{pmatrix} \begin{pmatrix} 0 & -\ii \\ \ii & 0 \end{pmatrix} \\
               &= \mathbb{1} \\
    \sigma_z^2 &= \begin{pmatrix} 1 & 0 \\ 0 & -1 \end{pmatrix}\begin{pmatrix} 1 & 0 \\ 0 & -1 \end{pmatrix} \\
               &= \mathbb{1} \\
    \sigma_x\sigma_y &= \begin{pmatrix} 0 & 1 \\ 1 & 0 \end{pmatrix}\begin{pmatrix} 0 & -\ii \\ \ii & 0 \end{pmatrix} \\
                     &= \begin{pmatrix} \ii & 0 \\ 0 & -\ii \end{pmatrix} \\
                     &= \ii\begin{pmatrix} 1 & 0 \\ 0 & -1 \end{pmatrix} \\
                     &= \ii\sigma_z
\end{align*}

% subsection Eigenschaften (end)

\subsection{Eigenzustände und Eigenwerte von $\sigma_z$} % (fold)

Zunächst die Eigenwerte:
\begin{align*}
    && \det\del{\sigma_z - \lambda_{\pm}\mathbb 1} &= 0 &&\\
\iff && \begin{vmatrix}1-\lambda_\pm & 0 \\ 0 & -1-\lambda_\pm \end{vmatrix} &= 0  \\
    \iff && \del{1-\lambda_\pm}\del{-1-\lambda_\pm} &= 0 \\
    \iff && \lambda_\pm^2 -1 &= 0. 
\end{align*}
Hieraus folgen direkt die Eigenwerte: $\lambda_+ =1$ und $ \lambda_- = -1$. Nun die Eigenzustände. Beginnen wir mit $\ket{+}$:
\begin{align*}
    && \sigma_z\ket{+} &= \lambda_+\ket{+} && \\
\iff && \begin{pmatrix}1 & 0 \\ 0 & -1 \end{pmatrix} \ket{+} &= 1\ket{+}. \\
\intertext{%
    Hieraus folgt offensichtlich
}
&& \ket{+} &= \begin{pmatrix}1\\0\end{pmatrix}.
\intertext{Nun $\ket{-}$:}
    && \sigma_z\ket{-} &= \lambda_-\ket{-} && \\
\iff && \begin{pmatrix}1 & 0 \\ 0 & -1 \end{pmatrix} \ket{-} &= -1\ket{-} \\
\implies && \ket{-} &= \begin{pmatrix}0\\1\end{pmatrix}
\end{align*}

\subsection{Eigenwerte und Eigenzustände von $\sigma_x$ und $\sigma_y$} % (fold)

\begin{align*}
    && \det\del{\sigma_x - \lambda_{\pm}\mathbb 1} &= 0 &&\\
\iff && \begin{vmatrix}-\lambda_\pm & 1 \\ 1 & -\lambda_\pm \end{vmatrix} &= 0  \\
    \iff && \lambda_\pm^2 -1 &= 0 \\
    \iff && \lambda_\pm^2  &= 1
    \intertext{%
        Es ergeben sich die gleichen Eigenwerte, wie für $\sigma_z$. Nun überprüfen wir, ob $\sigma_x$ auch die gleichen Eigenzustände besitzt:
    }
&&\sigma_x\begin{pmatrix}1\\0\end{pmatrix} &= \begin{pmatrix}0&1\\1&0\end{pmatrix}\begin{pmatrix}1\\0\end{pmatrix} \\
&&&= \begin{pmatrix}0\\1\end{pmatrix} \\
&&&\neq 1\cdot\begin{pmatrix}1\\0\end{pmatrix} \\
&&\sigma_x\begin{pmatrix}0\\1\end{pmatrix} &= \begin{pmatrix}0&1\\1&0\end{pmatrix} \begin{pmatrix}0\\1\end{pmatrix} \\
&&&= \begin{pmatrix}1\\0\end{pmatrix} \\
&&&\neq -1\cdot\begin{pmatrix}0\\1\end{pmatrix},
    \intertext{%
        $\sigma_x$ hat demnach andere Eigenzustände als $\sigma_z$.\newline
        Nun überprüfen wir $\sigma_y$:
    }
    && \det\del{\sigma_y - \lambda_{\pm}\mathbb 1} &= 0 &&\\
\iff && \begin{vmatrix}-\lambda_\pm & -\ii \\ \ii & -\lambda_\pm \end{vmatrix} &= 0  \\
    \iff && \lambda_\pm^2 -1 &= 0 \\
    \iff && \lambda_\pm^2  &= 1
    \intertext{%
        Es ergeben sich die gleichen Eigenwerte, wie für $\sigma_z$. Nun überprüfen wir, ob $\sigma_y$ auch die gleichen Eigenzustände besitzt:
    }
&&\sigma_y\begin{pmatrix}1\\0\end{pmatrix} &= \begin{pmatrix}0&-\ii\\\ii&0\end{pmatrix} \begin{pmatrix}1\\0\end{pmatrix} \\
&&&= \begin{pmatrix}0\\\ii\end{pmatrix} \\
&&&\neq 1\cdot\begin{pmatrix}1\\0\end{pmatrix} \\
&&\sigma_y\begin{pmatrix}0\\1\end{pmatrix} &= \begin{pmatrix}0&-\ii\\\ii&0\end{pmatrix} \begin{pmatrix}0\\1\end{pmatrix} \\
&&&= \begin{pmatrix}-\ii\\0\end{pmatrix} \\
&&&\neq -1\cdot\begin{pmatrix}1\\0\end{pmatrix},
\end{align*}
auch $\sigma_y$ besitzt also nicht die gleichen Eigenzustände, wie $\sigma_z$.

\end{document}
