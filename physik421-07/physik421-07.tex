% Für Seitenformatierung

\documentclass[DIV=15]{scrartcl}

% Zeilenumbrüche

\parindent 0pt
\parskip 6pt

% Für deutsche Buchstaben und Synthax

\usepackage[ngerman]{babel}

% Für Auflistung mit speziellen Aufzählungszeichen

\usepackage{paralist}

% zB für \del, \dif und andere Mathebefehle

\usepackage{amsmath}
\usepackage{commath}
\usepackage{amssymb}

% Für Literatur/bibliography

\usepackage[backend=biber , style=alphabetic , hyperref=true]{biblatex}

% Für \SIunit[]{} und \num in deutschem Stil

\usepackage[output-decimal-marker={,}]{siunitx}
\DeclareSIUnit\clight{\ensuremath{c}}

% Schriftart und encoding

\usepackage[utf8]{inputenc}
% Bitstream charter als default
\usepackage[charter, greekuppercase=italicized]{mathdesign}
% Lato, als sans default
\renewcommand{\sfdefault}{fla}

% Für \sfrac{}{}, also inline-frac

\usepackage{xfrac}

% Für Einbinden von pdf-Grafiken

\usepackage{graphicx}

% Umfließen von Bildern

\usepackage{floatflt}

% Für weitere Farben

\usepackage{color}

% Für Streichen von z.B. $\rightarrow$

\usepackage{centernot}

% Für Befehl \cancel{}

\usepackage{cancel}

% Für Links nach außen und innerhalb des Dokumentes

\usepackage{hyperref}

% Für Layout von Links

\hypersetup{
	citecolor=black,
	colorlinks=true,
	linkcolor=black,
	urlcolor=blue,
}

% Verschiedene Mathematik-Hilfen

\newcommand \e[1]{\cdot10^{#1}}
\newcommand\p{\partial}

\newcommand\half{\frac 12}
\newcommand\shalf{\sfrac12}

\newcommand\skp[2]{\left\langle#1,#2\right\rangle}
\newcommand\mw[1]{\left\langle#1\right\rangle}
\newcommand \eexp[1]{\mathrm{e}^{#1}}
\newcommand \dexp[1]{\exp\del{#1}}
\newcommand \dsin[1]{\sin\del{#1}}
\newcommand \dcos[1]{\cos\del{#1}}
\newcommand \dtan[1]{\tan\del{#1}}
\newcommand \darccos[1]{\arcos\del{#1}}
\newcommand \darcsin[1]{\arcsin\del{#1}}
\newcommand \darctan[1]{\arctan\del{#1}}

% Nabla und Kombinationen von Nabla

\renewcommand\div[1]{\skp{\nabla}{#1}}
\newcommand\rot{\nabla\times}
\newcommand\grad[1]{\nabla#1}
\newcommand\laplace{\triangle}
\newcommand\dalambert{\mathop{{}\Box}\nolimits}

%Für komplexe Zahlen

\newcommand \ii{\mathrm i}
\renewcommand{\Im}{\mathop{{}\mathrm{Im}}\nolimits}
\renewcommand{\Re}{\mathop{{}\mathrm{Re}}\nolimits}

%Für Bra-Ket-Notation

\newcommand\bra[1]{\left\langle#1\right|}
\newcommand\ket[1]{\left|#1\right\rangle}
\newcommand\braket[2]{\left\langle#1\left.\vphantom{#1 #2}\right|#2\right\rangle}
\newcommand\braopket[3]{\left\langle#1\left.\vphantom{#1 #2 #3}\right|#2\left.\vphantom{#1 #2 #3}\right|#3\right\rangle}



\newcounter{thezettel}
\setcounter{thezettel}{6}
\renewcommand\thesection{\arabic{thezettel}.\arabic{section}}

\newcommand\ccancel[2][black]{\renewcommand\CancelColor{\color{#1}}\cancel{#2}}


\title{physik421 - Übung \arabic{thezettel}}
\author{Lino Lemmer \\ \small{l2@uni-bonn.de} \and Frederike Schrödel \and Simon Schlepphorst\\ \small{s2@uni-bonn.de}}

\begin{document}
\maketitle

\section{Verschränkte Wellenfunktion}
\subsection{}

\section{Hamilton-Operator für Teilchen im externen $\vec E$ und $\vec B$ Feldern}
\subsection{}
Es ist die Lagrange-Funktion für ein geladenes Teilchen in externen elektromagnetischen Feldern gegeben:
\[
    L = \frac 12 m(\dot\vec x)^2-q(V-\dot\vec x\vec A)
\]
Um den kanonisch nonjugierten Impuls zu bestimmen nutze ich:
\[
    \vec P = \dpd{L}{\dot\vec x} = m\dot\vec x-q\vec A
\]

\subsection{}
Die dazu gehörige Hamilto-Funktion lautet:
\begin{align*}
    H &= \skp{\dot\vec q}{\vec P}-L \\
      &= \dot\vec x(m\dot\vec x+qA) -\frac 12 m(\dot\vec x)^2+q(V-\dot\vec xA) \\
      &= m\vec\dot x^2=\cancel{qA\vec\dot x}--\frac 12m\dot\vec x^2+qV-\cancel{q\dot\vec xA} \\
      &= \frac 12 m\vec\dot x^2+qV \\
      &= \frac{1}{2m}(m\vec\dot x)^2+qV \\
      &= \frac{1}{2m}(\vec p-q\vec A)+qV
\end{align*}

\subsection{}
Es soll gezeigt werden dass die Hamilton-Funktion die Gesamt- Energie des Systems beschreibt.
Die Gesamt-Energie erhalten wir durch:
\[
    E_\text{ges} = T+V
\]
Diese Formel entspricht in den meisten Fällen bereits der Hamilton-Funktion.
In diesen Fall sollte man berücksichtigen, dass die kinetische Energie gegeben ist durch $T= \frac 12 \vec\dot x^2$,
die potentielle Energie lautet allerdings eigentlich nur $V=qV$ da das Magnetfeld nur eine verschiebung verursacht.


\end{document}
