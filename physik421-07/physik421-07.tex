% Für Seitenformatierung

\documentclass[DIV=15]{scrartcl}

% Zeilenumbrüche

\parindent 0pt
\parskip 6pt

% Für deutsche Buchstaben und Synthax

\usepackage[ngerman]{babel}

% Für Auflistung mit speziellen Aufzählungszeichen

\usepackage{paralist}

% zB für \del, \dif und andere Mathebefehle

\usepackage{amsmath}
\usepackage{commath}
\usepackage{amssymb}

% Für Literatur/bibliography

\usepackage[backend=biber , style=alphabetic , hyperref=true]{biblatex}

% Für \SIunit[]{} und \num in deutschem Stil

\usepackage[output-decimal-marker={,}]{siunitx}
\DeclareSIUnit\clight{\ensuremath{c}}

% Schriftart und encoding

\usepackage[utf8]{inputenc}
% Bitstream charter als default
\usepackage[charter, greekuppercase=italicized]{mathdesign}
% Lato, als sans default
\renewcommand{\sfdefault}{fla}

% Für \sfrac{}{}, also inline-frac

\usepackage{xfrac}

% Für Einbinden von pdf-Grafiken

\usepackage{graphicx}

% Umfließen von Bildern

\usepackage{floatflt}

% Für weitere Farben

\usepackage{color}

% Für Streichen von z.B. $\rightarrow$

\usepackage{centernot}

% Für Befehl \cancel{}

\usepackage{cancel}

% Für Links nach außen und innerhalb des Dokumentes

\usepackage{hyperref}

% Für Layout von Links

\hypersetup{
	citecolor=black,
	colorlinks=true,
	linkcolor=black,
	urlcolor=blue,
}

% Verschiedene Mathematik-Hilfen

\newcommand \e[1]{\cdot10^{#1}}
\newcommand\p{\partial}

\newcommand\half{\frac 12}
\newcommand\shalf{\sfrac12}

\newcommand\skp[2]{\left\langle#1,#2\right\rangle}
\newcommand\mw[1]{\left\langle#1\right\rangle}
\newcommand \eexp[1]{\mathrm{e}^{#1}}
\newcommand \dexp[1]{\exp\del{#1}}
\newcommand \dsin[1]{\sin\del{#1}}
\newcommand \dcos[1]{\cos\del{#1}}
\newcommand \dtan[1]{\tan\del{#1}}
\newcommand \darccos[1]{\arcos\del{#1}}
\newcommand \darcsin[1]{\arcsin\del{#1}}
\newcommand \darctan[1]{\arctan\del{#1}}

% Nabla und Kombinationen von Nabla

\renewcommand\div[1]{\skp{\nabla}{#1}}
\newcommand\rot{\nabla\times}
\newcommand\grad[1]{\nabla#1}
\newcommand\laplace{\triangle}
\newcommand\dalambert{\mathop{{}\Box}\nolimits}

%Für komplexe Zahlen

\newcommand \ii{\mathrm i}
\renewcommand{\Im}{\mathop{{}\mathrm{Im}}\nolimits}
\renewcommand{\Re}{\mathop{{}\mathrm{Re}}\nolimits}

%Für Bra-Ket-Notation

\newcommand\bra[1]{\left\langle#1\right|}
\newcommand\ket[1]{\left|#1\right\rangle}
\newcommand\braket[2]{\left\langle#1\left.\vphantom{#1 #2}\right|#2\right\rangle}
\newcommand\braopket[3]{\left\langle#1\left.\vphantom{#1 #2 #3}\right|#2\left.\vphantom{#1 #2 #3}\right|#3\right\rangle}


\newcounter{thezettel}
\setcounter{thezettel}{7}
\renewcommand\thesection{\arabic{thezettel}.\arabic{section}}

\newcommand\ccancel[2][black]{\renewcommand\CancelColor{\color{#1}}\cancel{#2}}


\title{physik421 - Übung \arabic{thezettel}}
\author{Lino Lemmer \\ \small{l2@uni-bonn.de} \and Frederike Schrödel \and Simon Schlepphorst\\ \small{s2@uni-bonn.de}}

\begin{document}
\maketitle

\section{Verschränkte Wellenfunktion}

\subsection{}

\begin{align*}
    \psi &= \frac 1{\sqrt2} \del{\psi_{1,\text{L}}\psi_{2,\text{L}}-\psi_{1,\text{R}}\psi_{2,\text{R}}} \\
         &= \frac1{\sqrt2}
    \del{\frac{\del{\psi_{1,x}-\ii\psi_{1,y}}\del{\psi_{2,x}+\ii\psi_{2,y}}}2 -
    \frac{\del{\psi_{1,x}+\ii\psi_{1,y}}\del{\psi_{2,x}-\ii\psi_{2,y}}}2} \\
    &= \frac1{\sqrt8}\del{\psi_{1,x}\psi_{2_x} + \ii\psi_{1,x}\psi_{2,y} - \ii\psi_{2,x}\psi_{1,y} + \psi_{1,y}\psi_{2,y} - \psi_{1,x}\psi_{2,x} +  \ii\psi_{1,x}\psi_{2,y} - \ii\psi_{1,y}\psi_{2,x} - \psi_{1,y}\psi_{2,y} } \\
    &= \frac\ii{\sqrt2}\del{\psi_{1,x}\psi_{2,y}-\psi_{1,y}\psi_{2,x}}
\end{align*}

\subsection{}

Ergibt die Messung für Photon 1 eine lineare Polarisation in $x$-Richtung
können wir
\begin{enumerate}[(i)]
    \item
        aus der Darstellung mit linearer Basis sagen, dass Photon 2 in $y$-Richtung polarisiert ist.
    \item
        aus der Darstellung in zirkularen Basis nichts heraus finden.
\end{enumerate}

\subsection{}

Nun ergibt eine Messung eine lineare Polarisation in $x$-Richtung für Photon 1 und eine linkshändig zirkulare Polarisation für Photon 2.
\begin{enumerate}[(i)]
    \item
        Wird die Messung an Photon 1 zuerst durchgeführt, kollabiert die Wellenfunktion zu
        \[
            \psi_I = \psi_{1,x}\psi_{2,y}.
        \]
    \item
        Wird die Messung an Photon 2 zuerst durchgeführt, kollabiert die Wellenfunktion zu
        \[
            \psi_{II} = \psi_{1,\text{L}}\psi_{2,\text{L}}.
        \]
\end{enumerate}

\subsection{}

\begin{enumerate}[(i)]
    \item
        In linearer Basis lassen sich die beiden Wellenfunktionen aus der letzten Teilaufgabe als
    \begin{align*}
        \psi_I &= \psi_{1,x}\psi_{2,y}
        \intertext{%
            bzw. als
        }
        \psi_{II} &= \frac{\del{\psi_{1,x}-\ii\psi_{1,y}}\del{\psi_{2,x}+\ii\psi_{2,y}}}2 \\
        &= \half\del{\psi_{1,x}\psi_{2_x} + \ii\psi_{1,x}\psi_{2,y} - \ii\psi_{2,x}\psi_{1,y} + \psi_{1,y}\psi_{2,y}}
    \end{align*}
    ausdrücken.
    \item
        In zirkularer Basis lassen sich die beiden Wellenfunktionen als
        \begin{align*}
            \psi_I &= \frac\ii{2} \del{\psi_{1,\text{R}}-\psi_{1,\text{L}}} \del{\psi_{2,\text{R}}-\psi_{2,\text{L}}} \\
                 &= \frac \ii2\del{\psi_{1,\text{R}}\psi_{2,\text{R}} - \psi_{1,\text{R}}\psi_{2,\text{L}} - \psi_{1,\text{L}}\psi_{2,\text{R}} + \psi_{1,\text{L}}\psi_{2,\text{L}}}
            \intertext{%
                bzw. als
            }
            \psi_{II} &= \psi_{1,\text{L}}\psi_{2,\text{L}}
        \end{align*}
        darstellen.
    \item
        Als Mischbasis wähle ich für $\psi_1$ die lineare und für $\psi_2$ die
        zirkulare Basis. So erhalte ich für die beiden Wellenfunktionen nach
        der Messung
        \begin{align*}
            \psi_I &= \frac\ii{\sqrt2} \psi_{1,x}\del{\psi_{2,\text{R}}-\psi_{2,\text{L}}}
            \intertext{%
                und
            }
            \psi_{II} &= \frac1{\sqrt2} \del{\psi_{1,x} - \ii\psi_{1,y}}\psi_{2,\text{L}}.
            \intertext{%
                Setzt man nun in $\psi_I$ die Transformation von $\psi_{2,\text{R}}$ ein erhält man
            }
            \psi_I &= \frac{\ii}{\sqrt{2}} \psi_{1,x}\del{\frac{1}{\sqrt2}\del{\psi_{2,x}-\ii\psi_{2,y}} - \psi_{2,\text{L}}} \\
                   &= \frac\ii2\psi_{1,x}\psi_{2,x} + \half \psi_{1,x}\psi_{2,y} - \frac\ii{\sqrt2}\psi_{1,x}\psi_{2,\text{L}} \\
                   &\overset{?}{=} \frac1{\sqrt2} \del{\psi_{1,x} - \ii\psi_{1,y}}\psi_{2,\text{L}} \\
            &= \psi_{II}
        \end{align*}
\end{enumerate}


\section{Hamilton-Operator für Teilchen in externen $\vec E$ und $\vec B$ Feldern}

\section{Wellenfunktion mit minimaler Unschärfe}

\end{document}
